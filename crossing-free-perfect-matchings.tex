\documentclass[10pt, a4paper, twoside]{basestyle}

\usepackage{tikz}
\usetikzlibrary{calc}
\usetikzlibrary{decorations.pathmorphing}
\usepackage{tikz-qtree}
\usepackage{braids}

\usepackage[Mathematics]{semtex}

%%%% Shorthands.
\newcommand{\idest}{\emph{, i.e.\ }}

\DeclareMathOperator{\cells}{cells}
\DeclareMathOperator{\spc}{spc}

\newcommand*{\tn}[2]{\tikz[baseline,remember picture]\node[inner sep=0pt,anchor=base] (#1) {#2};}

%%%% Title and authors.

\title{%
\textdisplay{%
Crossing-Free Perfect Matchings%
}%
}
\author{Robin~Leroy}
\begin{document}
\maketitle

blurb

\section{Geometric graphs} 
blurb
\begin{definition}[geometric graph]
Given a set of points $P$ in the Euclidean plane $\R^2$,
a \emph{geometric graph} is a collection of straight line segments (edges)
whose endpoints are elements of $P$.

It can be described as a simple graph (in the combinatorial sense)
on the vertices $P$, where the edge $\set{v,w}$ corresponds to the segment joining
$v$ and $w$.
\end{definition}
\begin{definition}[crossing-free geometric graph]
A geometric graph is \emph{crossing-free} if no two edges share points other than
their endpoints; it is called \emph{crossing} otherwise.

Note that this implies that the corresponding simple graph is planar, and that the
geometric graph is a plane embedding.
\end{definition}
\begin{definition}[triangulation]
A \emph{triangulation} is a maximal crossing-free geometric graph, that is, a
geometric graph such that for all $v$ and $w$ in $P$ that are not joined by a
segment, adding the segment joining $v$ and $w$ would result in a crossing
geometric graph.

Note that the faces (in the sense of plane graphs) formed by a triangulation are
all triangles, with the possible exception of the outer face (thus this definition
is \emph{not} equivalent to that of a triangulation of the $2$-sphere).
\end{definition}

Since a geometric graph corresponds to a simple graph on the underlying point set,
we can also look at geometric graphs that belong special classes of simple graphs.
\begin{definition}[crossing-free matching]
A crossing-free geometric graph is a \emph{crossing-free matching} if it is a matching
as a simple graph on the vertices $P$.
\end{definition}
\begin{definition}[crossing-free perfect matching]
A \emph{crossing-free perfect matching} is a crossing-free geometric graph which is
perfect matching as a simple graph on the vertices $P$.
\end{definition}
\section{Bounds and asymptotics}
There is interest in statements regarding the number of possible geometric graphs in
in the aforementioned classes; evidently, that number would depend on the choice of
the point set $P$, so instead one is interested in bounds on that number depending
on the cardinality $\Cardinality P$, and possibly restricting $P$ so that it satisfies
certain properties.

In general, if $g\of P$ is the number of geometric graphs of a certain sort on the point
set $P$, we will look for lower bounds $l$ and upper bounds $u$ of the form\[
\forall n\in \N, \forall P \text{ such that } \Cardinality P = n,
l \of n \leq g \of n \leq u \of n \text,\]
where the $P$ runs over all point sets that satisfy the relevant properties.

Alternatively, we may be interested in asymptotics on such $l$s and $u$s.
\section{Convex point sets}
\section{Crossing-free perfect matchings}

[Somewhere, define left-right matching, "general position wrt the horizontal",
"numbered from left to right"; prove that a matching yields a WFBE;
when talking about matchings we will talk about "the edge of a point"]

\section{Brackets expressions and an optimal lower bound}
This argument is due to E.~Welzl [cite paper to appear, is there a preprint?].

\begin{theorem}
Let $P$ be a point set of size $2n$ in general position with respect to the horizontal,
numbered from left to right, and let $B$ be a well-formed bracket expression of size $2n$.
Then there exists a crossing-free perfect matching such that the $k$th point of $P$ is a left
endpoint if and only if the $k$th bracket of $B$ is an opening bracket.
\begin{proof}
Let $m_0$ be a perfect matching on $P$ consistent with $B$. This is always possible, for instance,
parsing the bracket expression, match the point corresponding with an opening parenthesis
to the point corresponding with the matching closing parenthesis.

Define $l(m)$ for a perfect matching $m$ on $P$ to be the sum of the lengths of the edges of $m$.

\marginfig[\label{figUncrossing}Uncrossing in a left-right perfect matching.]{TODO}
Then, repeat the following procedure, starting at $k=0$.
If there is no crossing in $m_k$, we have found a perfect matching with the desired properties.
If there is a crossing, let $a$, $b$, $c$, and $d$ be the points involved, so that the edge
$ab$ crosses the edge $cd$. Remove these edges, and replace them by $ad$ and $cb$
(thus ``uncrossing'' them). This yields another perfect matching $m_{k+1}$. By the triangle
inequality (see figure \ref{figUncrossing}), $l(m_{k+1})<l(m_k)$.

If this did not terminate, it would yield a sequence $m$ of crossing perfect matchings on $P$ on
which $l$ is strictly decreasing, thus an infinite sequence of graphs on $P$.
Since there are only finitely many graphs on $P$, this is a contradiction, so we eventually find a
crossing-free perfect matching.
\end{proof}
\end{theorem}
This immediately yields a lower bound, since there are $C_n$ well-formed bracket expressions
of size $2n$.
\begin{corollary}
Let $P$ be a point set of size $2n$ in general position. There are at least $C_n$ distinct
crossing-free perfect matchings on $P$.
\end{corollary}
Moreover, this lower bound is optimal, since it is attained if $P$ is in convex position.

[TODO something about the general idea of proving upper bounds for left-right perfect matchings or
classes thereof to get an upper bound on perfect matchings]

\section{Matchings across a line}
TODO rewrite this with $n$ points in total, and maybe name $n/2$, otherwise this is going to be inconsistent
with subsequent sections and confusing.

Again we consider $2n$ points in general position with respect to the horizontal.

The left-right matchings corresponding to brackets expressions with $n$ opening brackets followed
by $n$ closing brackets, $\langle \dotsb \langle \rangle \dotsb \rangle$, are called \emph{matchings
across a line}. Indeed, any segment in such a matching will cross any vertical line that separates
the left-points from the right-points. [FIGURE]

The following result was shown by Micha Sharir and Emo Welzl in 2006 [CITATION HERE].
\begin{theorem}[Sharir--Welzl]
There are at most $C_n^2$ crossing-free perfect matchings across a line on $2n$ points in general position with
respect to the horizontal.
\begin{proof}
Pick a vertical line that separates the left-points from the right-points; we will call it \emph{the vertical line}.
Further, let us call set of left-points $L$ and the set of right-points $R$.

A perfect matching across a line is uniquely defined by a bijection $\FunctionSpec \gm L R$ from
the left-points to the right-points.
Now, number the intersections between the edges of the perfect matching and the vertical
line from top to bottom. This yield a numbering $\FunctionSpec \gn E {[n]}$ of the edges.

Define $\gl\of l \DefineAs \gn\of{e_l}$ mapping the number of a left-point to the intersection number of its
edge, and similarly $\gr\of r \DefineAs \gn\of{e_r}$ for the right-points. We have $\gm = \gl \gr^{-1}$.

The permutations $\gl$ (respectively $\gr$) determine the order in which the left points (respectively right points)
reach the vertical line.

The idea of the proof is as follows: if the matching is crossing-free, we will show that
$\gl$ and $\gr$ have to be in sets of size $C_n$, thus that $\gm = \gl \gr^{-1}$ can take at most
$C_n^2$ values\idest that there can be at most $C_n^2$ perfect matchings across a line.

Since we are going to reuse these concepts in subsequent proofs, we will formalize and name the properties of
$\gl$ and $\gr$ that we will consider. A bijection from a set of left-points $\gL$ to $[\Cardinality \gL]$
that can be constructed by numbering from top to bottom the intersections of the edges incident to $\gL$ with a
vertical line to the right of $\gL$ is called a \emph{crossing-free left-alignment of $\gL$}. Correspondingly, for
right-points, we define a \emph{crossing-free right-alignment}.
\marginfig[A crossing-free left-alignment]{TODO}

If we have a crossing-free perfect matching across a line, then $\gl$ as constructed above is a crossing-free
left-alignment of $L$, and $\gr$ is a crossing-free right-alignment of $R$.

\begin{lemma}
There are at most $C_n$ crossing-free left-alignments of $n$ points.
\begin{proof}
Let $l$ be the leftmost point of a set of $L$ of $n$ left-points, and let $\gl$ be a crossing-free
left-alignment of $L$.

$\gl(l)$\idest the index of the crossing of the vertical line the edge $e_l$ of the leftmost point,
is equal to one plus the number of points that are above $e_l$. Indeed, the edges of points above $e_l$ must
themselves reach the vertical line above $e_l$, otherwise they would cross $e_l$, and correspondingly for
points below $e_l$, so that there are as many edges reaching the vertical line below $e_l$ as there are points
below $e_l$.

Moreover, as the oriented angle between $e_l$ and the horizontal increases, points are only added to the
set of points below $e_l$, so that choosing the number of points below $e_l$ determines the sets of points
below and above $e_l$.

Further, since points above $e_l$ must reach the line above $e_l$ and correspondingly for points below,
if the point $p$ is above $e_l$, then $\gl(p) < \gl(l)$, and if it is below, $\gl(p) > \gl(l)$. Thus,
$\gl$ restricted do the points above $e_l$ is a crossing-free left-alignment of the $\gl(l) - 1$ points above
$e_l$, and $\gl-\gl(l)$ is a crossing-free left-alignment of the $n - \gl(l)$ points below $e_l$.

Thus, $\gl$ is determined by the choice of $i \DefineAs \gl(l) - 1$ and crossing-free left-alignments
of $i$ and $n-i-i$ points. It follows that if $\varpi_k$ is a bound for the number of crossing-free
left-alignments of $k$ points when $k < n$,
we can give a bound on the number of crossing-free left-alignments of $n$ points,
\[\varpi_n\DefineAs\sum{i=0}[k]\varpi_{i}\varpi_{n-i-1}\text.\]
We can start the recurrence with $\varpi_0 = 1$; this is the recurrence for the Catalan numbers, thus
$\varpi_k=C_k$.
\end{proof}
\end{lemma}
The same result holds for right-alignments, completing the proof.
\end{proof}
\end{theorem}
\section{Analysing the overcounting in the upper bound for matchings across a line}
\section{Highly convex matchings across a line}
\section{Three changes of bracket direction}
We now consider left-right matchings corresponding to bracket expressions which have three
changes of bracket directions\idest bracket expressions of the form
$\langle\dotsb\langle\rangle\dotsb\rangle\langle\dotsb\langle\rangle\dotsb\rangle$,
$k$ opening brackets, $l$ closing brackets, $q$ opening brackets, $p$ closing brackets, where
$k-l=p-q\geq 0$ and $k+l+p+q\DefinitionOf n$.
We will call the bracket expression of that form $\langle^k\rangle^l\langle^q\rangle^p$.

In term of points, this means that four sets can be separated by vertical lines, from left
to right, $k$ left-points forming the set $K$, $l$ right-points forming $L$, $q$ left-points
forming $Q$, ad $p$ right-points forming $P$.
We pick a vertical line separating $K$ and $L$ and call it \emph{the left line}, and we pick
a vertical line separating $P$ and $Q$ and call it \emph{the right line}.

Given a crossing-free perfect matching on those points, numbering from top to bottom the
intersections between edges incident to the points of $K$ and and the left line,
we get a crossing-free left-alignment of $K$. $k-l$ of the $k$ edges on this vertical
line are incident to points in $P$; the other $l$ are incident to points in $L$.
Numbering those $l$ edges yields a crossing-free right-alignment $\gl$ of $L$.

Similarly on the right side, we get a crossing-free right-alignment of $\gp$ of $P$,
numbering the subset of edges incident to a point in $P$ and a point in $Q$, we get a
left-alignment $\gq$ of $q$.
\marginfig[Four crossing-free alignments.]{TODO}

The matching is uniquely determined by $\gk$, $\gp$, the choice of
the $k-l$ among $k$ points and $k-l=p-q$ among $p$ points that get matched to each other,
and by $\gl$ and $\gq$, which gives the following bound for the number of these matchings:\[
\Cardinality{f_P^{-1}\of{\set{\langle^k\rangle^l\langle^q\rangle^p}}}
\leq C_k \binom k l C_l C_q \binom p q C_p\text.\]

[TODO give the asymptotics of this here; it's not so good, but it motivates the improvements below]

\subsection{Improving the binomial bound}
We can however improve upon that bound: indeed, once $\gk$, $\gp$, and the set of points
of $K$ matched to $P$ are fixed, the edges of the matching that cross both the left line
and the right line---let us call these \emph{long edges}---are determined. As a result,
the region between the left line and the
right line is partitioned in trapezoidal cells, and the portion of any edge from $K$ to $L$ and
from $P$ to $Q$ that lies between the left line and the right line is confined to a
single of those cells. It follows that $\gl$ is composed of crossing-free
right-alignments the of subsets of $L$ separated by the long edges, and similarly for $\gq$
with subsets of $Q$.
\marginfig[Splitting the middle two crossing-free alignments.]{TODO}

Let us look at the edges crossing the left line (the same argument applies to the right line),
numbered from top to bottom: $k-l$ of those are
long edges, let $S \Subset [k]$ be their numbers; in between two long edges, above the first
long edge, and below the last one, we have the edges that define the crossing-free right-alignments
that make up $\gl$. It follows that a crossing-free right-alignment of $m$ points that makes up
$\gl$ corresponds to a maximal sequence of $m$ consecutive elements of $[k] \setminus S$.

We will call the set of maximal sequences of consecutive elements of $S'$ the \emph{cells} of
$S'$, written $\cells\of{S'}$.

Then, we can improve the $\binom k l C_l$ factor in the bound (in which the binomial comes from
the choice of the long edges amongst the $k$ edges on the left line, and the Catalan number comes
from the choice of $\gl$), summing over the choices of the long edges (and thus of $S$ above).
The improved factor becomes\[
\spc\of{k, l}\DefineAs
\sum{S\in\binom{[k]}{k-l}} \quad \prod{c\in\cells\of{[k] \setminus S}}C_{\Cardinality c} =
\sum{S'\in\binom{[k]}{l}} \quad \prod{c\in\cells\of{S'}}C_{\Cardinality c}\text,
\]
and the overall bound becomes\[
C_k\spc\of{k, l}\spc\of{p, q}C_p\text.
\]

\subsection{A recurrence}
In order to compute $\spc$ efficiently, and eventually, get its asymptotics, it is useful to get rid
of the $\cells$ function. We can express $\spc$ as a recurrence instead.
First, we note that $\spc\of{k,k} = C_k$: there is only one summand, $S'$ is the whole set, so
it has only one cell, namely itself.
Otherwise, $k-l\geq 1$; in the sum over the $S$, consider the greatest element $j$ of $S$, which is at least
$k-l$, and split the sum over that,
\[\spc\of{k, l} = \sum{j=k-l}[k]\quad
\sum{\substack{S\in\binom{[k]}{k-l} \\ j=\max S}} \quad
\prod{c\in\cells\of{[k] \setminus S}}C_{\Cardinality c}\text.
\]
For fixed $j$, all summands (of the sum over $S$) will have a factor
with $c=\set{j+1,\dotsc,k}$, and thus a factor of
$C_{k-j}$. Factoring out this $C_{k-j}$, we get
\[\spc\of{k, l} = \sum{j=k-l}[k]
C_{k-j} \quad
\sum{\substack{S\in\binom{[k]}{k-l} \\ j=\max S}} \quad
\prod{\substack{c\in\cells\of{[k] \setminus S} \\ c\neq\set{j+1,\dotsc,k}}}
    C_{\Cardinality c}\text.
\]
Now, note that choosing a subset $S$ of $[k]$ of size $k-l$ whose maximum is $j$ is equivalent to choosing
a subset $s$ of $[j-1]$ of size $k-l - 1$, where $S=s\Union\set{j}$. Moreover, the cells of $[k]\setminus S$ other
than $\set{j+1,\dotsc,k}$ are exactly the cells of $[j-1]\setminus s$, thus
\[\spc\of{k, l} = \sum{j=k-l}[k]
C_{k-j} \quad
\sum{s\in\binom{[j-1]}{k-l - 1}} \quad
\prod{c\in\cells\of{[j-1]\setminus s}}
    C_{\Cardinality c}\text.
\]
By definition of $\spc$, this means
\[\spc\of{k, l} = \sum{j=k-l}[k]
C_{k-j}
\spc\of{j-1,l+j-k}\text.
\]
Rewriting this as a sum over $i\DefineAs k-j$, this gives us the following recurrence for $\spc$:
\begin{align}\spc\of{k, l} &= \sum{i=0}[l]
C_{i}
\spc\of{k-i-1,l-i} &\text{for $l<k$,} \label{spcRecurrence1}\\
\spc\of{k, k} &= C_k \label{spckk}\text.
\end{align}\marginfig[The first few values of $\spc$; $k$ vertically from $0$ to $5$,
$l$ horizontally from $0$ to $k$. The $l+1$ summands in the recurrence (\ref{spcRecurrence1})
with $k = 6$ and $l = 3$ are highlighted.]{
$\begin{array}{r r r r r r}
1 \\
{\color{red}1} & 1 \\
1 & {\color{red}2} &  2 \\
1 & 3 &  {\color{red}5} & 5 \\
1 & 4 &  9 & \tn{lastSummand}{\color{red}$14$} & 14 \\
1 & 5 & 14 & \tn{result}{$28$} & 42 & 42
\end{array}$
\tikz[remember picture,overlay]
    \path[->] (lastSummand.south) edge (result.north -| lastSummand.south);
}
\subsection{A better recurrence}
We can now turn this recurrence into a simpler recurrence, which we will prove by recurrence.

Since $\spc$ has not been formally defined for negative arguments, we extend the definition
with $\spc\of{k,l}=0$ for $l<0$; this is consistent with the definition, since it yields a sum over
subsets $S\Subset [k]$ bigger than $k$, and it yields an empty sum in the recurrence we just derived.
In addition to that, we also let $\spc\of{k,l} = 0$ when $l > k$.

Note that for $k - 1 = l\geq 0$, we have\begin{align}
\spc\of{k, l}
= \sum{i=0}[l]C_{i}\spc\of{k-i-1,l-i} = \sum{i=0}[l]C_{i}\spc\of{l-i,l-i} 
&= \sum{i=0}[l]C_{i}C_{l-i} \nonumber \\
&= C_{l+1}\text,\label{spckkminusone}
\end{align}
and thus 
\begin{equation}
\spc\of{k, l} = \spc\of{k-1, l} + \spc\of{k, l-1} = C_k \text{ for $k = l > 0$.}
\end{equation}
Further, for $k - 1 = l$, we get
\begin{align}
\spc\of{k, l}
&= \sum{i=0}[l]C_{i}C_{l-i} = C_l + \sum{i=0}[l-1]C_{i}C_{((l-1)-i)+1} \nonumber\\
\intertext{applying (\ref{spckk}) on the left and (\ref{spckkminusone}) on the right,}
&= \spc\of{k-1, l} + \sum{i=0}[l-1]C_{i}\spc\of{k-i-1,(l-1)-i} \nonumber\\
&= \spc\of{k-1, l} + \spc\of{k,l-1}\text.
\end{align}

Now let $k - 1 > l \geq 0$.
Assume $\spc\of{k',l'}=\spc\of{k'-1,l'}+\spc\of{k',l'-1}$ for $0\leq l'<k'<k$.
Then we can apply this assumption to the summands of $\spc\of{k, l}$:
\marginfig[The recurrence applied to the sum, yielding two sums, proving the recurrence.]{TODO}
\begin{align}
\spc\of{k, l} &= \sum{i=0}[l]
C_{i}
\spc\of{k-i-1,l-i} \nonumber \\
&= \sum{i=0}[l]
C_{i}
\pa{\spc\of{k-i-2,l-i}+\spc\of{k-i-1,l-i-1}} \nonumber \\
&= \sum{i=0}[l]
C_{i}\spc\of{k-i-2,l-i}
+\sum{i=0}[l]C_{i}\spc\of{k-i-1,l-i-1} \nonumber \\
&= \sum{i=0}[l]
C_{i}\spc\of{(k-1)-i-1,l-i}
+\sum{i=0}[l-1]C_{i}\spc\of{k-i-1,(l-1)-i} \\&\phantom{=}\qquad + C_{l}\spc\of{k-l-1,-1}\text, \nonumber \\
\intertext{so, substituting the recurrence for $\spc$,}
&= \spc\of{k-1, l} + \spc\of{k,l-1}\text.
\end{align}
We thus have the following recurrence for $\spc$:
\begin{align}
\spc\of{k,l} &= \spc\of{k-1, l} + \spc\of{k,l-1} & \text{for $k\geq 1$, $0\leq l \leq k$,}\\
\spc\of{k,l} &= 0 & \text{for $l < 0$ or $l > k$,}\\
\spc\of{0,0} &= 1\text.
\end{align}
This is the recurrence defining the Catalan triangle [CITATIONS], and its solution is known, namely [CITATION]
\begin{equation}
\spc\of{k,l} = \frac
{\Factorial{\pa{k+l}} \Factorial{\pa{k-l+1}}}
{\Factorial l \Factorial{\pa{k+1}}}\text.
\end{equation}
\subsubsection{A combinatorial interpretation of the first recurrence}
\subsection{Asymptotics}
We can now use the expression for $\spc$ to study the asymptotics of the bound.

\clearpage
\nocite{*}
\bibliography{crossing-free-perfect-matchings}
\bibliographystyle{plain}

\end{document}
