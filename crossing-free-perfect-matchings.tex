\documentclass[10pt, a4paper, twoside]{basestyle}

\usepackage{tikz}
\usetikzlibrary{calc}
\usetikzlibrary{decorations.pathmorphing}
\usepackage{tikz-qtree}
\usepackage{braids}

\usepackage[Mathematics]{semtex}

%%%% Shorthands.
\newcommand\CAT[1]{\operatorname{CAT}\pa{#1}}
\newcommand{\idest}{\emph{, i.e.\ }}

%%%% Title and authors.

\title{%
\textdisplay{%
Crossing-Free Perfect Matchings%
}%
}
\author{Robin~Leroy}
\begin{document}
\maketitle

blurb

\section{Geometric graphs} 
blurb
\begin{definition}[geometric graph]
Given a set of points $P$ in the Euclidean plane $\R^2$,
a \emph{geometric graph} is a collection of straight line segments (edges)
whose endpoints are elements of $P$.

It can be described as a simple graph (in the combinatorial sense)
on the vertices $P$, where the edge $\set{v,w}$ corresponds to the segment joining
$v$ and $w$.
\end{definition}
\begin{definition}[crossing-free geometric graph]
A geometric graph is \emph{crossing-free} if no two edges share points other than
their endpoints; it is called \emph{crossing} otherwise.

Note that this implies that the corresponding simple graph is planar, and that the
geometric graph is a plane embedding.
\end{definition}
\begin{definition}[triangulation]
A \emph{triangulation} is a maximal crossing-free geometric graph, that is, a
geometric graph such that for all $v$ and $w$ in $P$ that are not joined by a
segment, adding the segment joining $v$ and $w$ would result in a crossing
geometric graph.

Note that the faces (in the sense of plane graphs) formed by a triangulation are
all triangles, with the possible exception of the outer face (thus this definition
is \emph{not} equivalent to that of a triangulation of the $2$-sphere).
\end{definition}

Since a geometric graph corresponds to a simple graph on the underlying point set,
we can also look at geometric graphs that belong special classes of simple graphs.
\begin{definition}[crossing-free matching]
A crossing-free geometric graph is a \emph{crossing-free matching} if it is a matching
as a simple graph on the vertices $P$.
\end{definition}
\begin{definition}[crossing-free perfect matching]
A \emph{crossing-free perfect matching} is a crossing-free geometric graph which is
perfect matching as a simple graph on the vertices $P$.
\end{definition}
\section{Bounds and asymptotics}
There is interest in statements regarding the number of possible geometric graphs in
in the aforementioned classes; evidently, that number would depend on the choice of
the point set $P$, so instead one is interested in bounds on that number depending
on the cardinality $\Cardinality P$, and possibly restricting $P$ so that it satisfies
certain properties.

In general, if $g\of P$ is the number of geometric graphs of a certain sort on the point
set $P$, we will look for lower bounds $l$ and upper bounds $u$ of the form\[
\forall n\in \N, \forall P \text{ such that } \Cardinality P = n,
l \of n \leq g \of n \leq u \of n \text,\]
where the $P$ runs over all point sets that satisfy the relevant properties.

Alternatively, we may be interested in asymptotics on such $l$s and $u$s.
\section{Convex point sets}
\section{Crossing-free perfect matchings}

[Somewhere, define left-right matching, "general position wrt the horizontal",
"numbered from left to right"; prove that a matching yields a WFBE;
when talking about matchings we will talk about "the edge of a point"]

\section{Brackets expressions and an optimal lower bound}
This argument is due to E.~Welzl [cite paper to appear, is there a preprint?].

\begin{theorem}
Let $P$ be a point set of size $2n$ in general position with respect to the horizontal,
numbered from left to right, and let $B$ be a well-formed bracket expression of size $2n$.
Then there exists a crossing-free perfect matching such that the $k$th point of $P$ is a left
endpoint if and only if the $k$th bracket of $B$ is an opening bracket.
\begin{proof}
Let $m_0$ be a perfect matching on $P$ consistent with $B$. This is always possible, for instance,
parsing the bracket expression, match the point corresponding with an opening parenthesis
to the point corresponding with the matching closing parenthesis.

Define $l(m)$ for a perfect matching $m$ on $P$ to be the sum of the lengths of the edges of $m$.

Then, repeat the following procedure, starting at $k=0$.
If there is no crossing in $m_k$, we have found a perfect matching with the desired properties.
If there is a crossing, let $a$, $b$, $c$, and $d$ be the points involved, so that the edge
$ab$ crosses the edge $cd$. Remove these edges, and replace them by $ad$ and $cb$
(thus ``uncrossing'' them). This yields another perfect matching $m_{k+1}$. By the triangle
inequality (see figure [TODO FIGURE]), $l(m_{k+1})<l(m_k)$. 

If this did not terminate, it would yield a sequence $m$ of crossing perfect matchings on $P$ on
which $l$ is strictly decreasing, thus an infinite sequence of graphs on $P$.
Since there are only finitely many graphs on $P$, this is a contradiction, so we eventually find a
crossing-free perfect matching.
\end{proof}
\end{theorem}
This immediately yields a lower bound, since there are $C_n$ well-formed bracket expressions
of size $2n$.
\begin{corollary}
Let $P$ be a point set of size $2n$ in general position. There are at least $C_n$ distinct
crossing-free perfect matchings on $P$.
\end{corollary}
Moreover, this lower bound is optimal, since it is attained if $P$ is in convex position.

[TODO something about the general idea of proving upper bounds for left-right perfect matchings or
classes thereof to get an upper bound on perfect matchings]

\section{Matchings across a line}
Again we consider $2n$ points in general position with respect to the horizontal.

The left-right matchings corresponding to brackets expressions with $n$ opening brackets followed
by $n$ closing brackets, $\langle \dotsb \langle \rangle \dotsb \rangle$, are called \emph{matchings
across a line}. Indeed, any segment in such a matching will cross any vertical line that separates
the left-points from the right-points. [FIGURE]

The following result was shown by Micha Sharir and Emo Welzl in 2006 [CITATION HERE].
\begin{theorem}[Sharir--Welzl]
There are at most $C_n^2$ crossing-free perfect matchings across a line on $2n$ points in general position with
respect to the horizontal.
\begin{proof}
Pick a vertical line that separates the left-points from the right-points; we will call it `the vertical line'.

Number the left-points from left to right, and the right-points from right to left.
A perfect matching across a line is uniquely defined by a bijection from the left-points to the right-points,
and thus, via the numbering, by a bijection from $[n]$ to itself\idest a permutation $\gm \in S_n$.
Now, number from bottom to top the intersections between the segments of the perfect matching and the vertical
line.
This yields two permutations, the left permutation $\gl$ mapping the number of a left-point to its intersection number,
and the right permutation $gr$ mapping the number of a right-point to its intersection number. [FIGURE]
Moreover, we have $\gm = \gl \gr^{-1}$.

The permutations $\gl$ (respectively $\gr$) determine the order in which the left points (respectively right points)
reach the vertical line.

If the matching is crossing-free, we will show that $\gl$ and $\gr$ have to be in sets of size $C_n$, and thus
that $\gm = \gl \gr^{-1}$ can take at most $C_n^2$ values\idest that there can be at most $C_n^2$ perfect matchings
across a line.

[REWORD THE PART BELOW, NAME THE SET TO WHICH $\gl$ BELONGS]

To constrain $\gl$ (the same argument holds for $\gr$ with left and right reversed), we look only at the left-points
and the vertical line, and we note that the value of
$\gl(1)$\idest the index of the crossing of the vertical line the edge $e_1$ of the leftmost point,
is equal to one plus the number of points that are below $e_1$. Indeed, the edges of points above $e_1$ must
themselves reach the vertical line above $e_1$, otherwise they would cross $e_1$, and correspondingly for
points below $e_1$, so that there are as many edges reaching the vertical line below $e_1$ as there are points
below $e_1$.

Moreover, as the angle of $e_1$ increases, points are only added to the set of points below $e_1$, so that choosing
the number of points below $e_1$ determines the sets of points below and above $e_1$.

Additionally, since points above $e_1$ must reach the line above $e_1$ and correspondingly for points below,
if the point numbered $k$ is above $e_1$, then $\gl(k) > \gl(1)$, and if it is below, $\gl(k) < \gl(1)$.
$\gl$ restricted do the points above and below $e_1$ thus yields bijections from the points
below $e_1$ to $[1, \gl(1) - 1] \Intersection \Z$, and from the points above to $[\gl(1) + 1, n]$.
Renumbering, those are permutations on $\gl(1) - 1$ and $n-\gl(1)$ elements, respectively, and they are
defined as $\gl$ itself was, by the edges.

Thus, having chosen $i = \gl(1) - 1$ we can choose independently in the same way the permutation of the $\gl(1) - 1$,
points below and that of the $n - \gl(1)$ points above $e_1$; this yields a recurrence for an upper bound on
the number $\varpi_n$ of permutations $\gl$ on $n$ points that can be obtained without crossings,
\[\varpi_k\leq\sum{i=0}[k]\varpi_{i}\varpi_{n-i-1}\text,\]
where $\varpi_0 = 1$. With equality, this is the recurrence for the Catalan numbers, thus
$\varpi_k\leq C_k$.
\end{proof}
\end{theorem}

\section{Overcounting in the upper bounds for matchings across a line}
\section{Highly convex matchings across a line}

\clearpage
\nocite{*}
\bibliography{crossing-free-perfect-matchings}
\bibliographystyle{plain}

\end{document}
