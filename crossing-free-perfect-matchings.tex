\documentclass[10pt, a4paper, twoside]{basestyle}

\usepackage{tikz}
\usetikzlibrary{calc}
\usetikzlibrary{decorations.pathmorphing}
\usetikzlibrary{intersections}
\usepackage{tikz-qtree}
\usepackage{braids}

\usepackage[Mathematics]{semtex}

%%%% Shorthands.
\newcommand{\idest}{\emph{, i.e.\ }}

\DeclareMathOperator{\cells}{cells}
\DeclareMathOperator{\spc}{spc}
\DeclareMathOperator{\PM}{PM}
\DeclareMathOperator{\CFPM}{CFPM}

\newcommand*{\tn}[2]{\tikz[baseline,remember picture]\node[inner sep=0pt,anchor=base] (#1) {#2};}

%%%% Title and authors.

\title{%
\textdisplay{%
Crossing-Free Perfect Matchings%
}%
}
\author{Robin~Leroy}
\begin{document}
\maketitle

blurb

\section{Geometric graphs} 
blurb
\begin{definition}[geometric graph]
Given a set of points $P$ in the Euclidean plane $\R^2$,
a \emph{geometric graph} is a collection of straight line segments (edges)
whose endpoints are elements of $P$.

It can be described as a simple graph (in the combinatorial sense)
on the vertices $P$, where the edge $\set{v,w}$ corresponds to the segment joining
$v$ and $w$.
\end{definition}
\begin{definition}[crossing-free geometric graph]
A geometric graph is \emph{crossing-free} if no two edges share points other than
their endpoints; it is called \emph{crossing} otherwise.

Note that this implies that the corresponding simple graph is planar, and that the
geometric graph is a plane embedding.
\end{definition}
\begin{definition}[triangulation]
A \emph{triangulation} is a maximal crossing-free geometric graph, that is, a
geometric graph such that for all $v$ and $w$ in $P$ that are not joined by a
segment, adding the segment joining $v$ and $w$ would result in a crossing
geometric graph.

Note that the faces (in the sense of plane graphs) formed by a triangulation are
all triangles, with the possible exception of the outer face (thus this definition
is \emph{not} equivalent to that of a triangulation of the $2$-sphere).
\end{definition}

Since a geometric graph corresponds to a simple graph on the underlying point set,
we can also look at geometric graphs that belong special classes of simple graphs.
\begin{definition}[crossing-free matching]
A crossing-free geometric graph is a \emph{crossing-free matching} if it is a matching
as a simple graph on the vertices $P$.
\end{definition}
\begin{definition}[crossing-free perfect matching]
A \emph{crossing-free perfect matching} is a crossing-free geometric graph which is
perfect matching as a simple graph on the vertices $P$.
\end{definition}
\section{Bounds and asymptotics}
There is interest in statements regarding the number of possible geometric graphs in
in the aforementioned classes; evidently, that number would depend on the choice of
the point set $P$, so instead one is interested in bounds on that number depending
on the cardinality $\Cardinality P$, and possibly restricting $P$ so that it satisfies
certain properties.

In general, if $g\of P$ is the number of geometric graphs of a certain sort on the point
set $P$, we will look for lower bounds $l$ and upper bounds $u$ of the form\[
\forall n\in \N, \forall P \text{ such that } \Cardinality P = n,
l \of n \leq g \of n \leq u \of n \text,\]
where the $P$ runs over all point sets that satisfy the relevant properties.

In addition, we may be interested in asymptotics on such $l$s and $u$s; since these bounds are
often exponential, we tend to ignore polynomial factors; we will thus say that\[
f \of n \preccurlyeq u \of n
\]
if $f\of n \leq p\of n u \of n$ for some polynomially-bounded $p$.
\section{Convex point sets}
[talk about the number of triangulations and crossing-free perfect matchings; this is a good place
to introduce the recurrence for Catalan numbers too]

\section{Crossing-free perfect matchings}
We will call $\PM_P$ the set of perfect matchings on the point set $P$, and $\CFPM_P$ the set
of crossing-free perfect matchings on the point set $P$.
[TODO cite existing results]

\section{Left-right perfect matchings and bracket expressions}
A point set in the plane is \emph{in general position} if no three points are aligned.

We say that a point set is \emph{in general position with respect to the horizontal} if it
is in general position and no two points lie on a vertical line. Note that any point set in
general position can be put in general position with respect to the horizontal by an arbitrarily
small rotation. Moreover, note that points in general position with respect to the horizontal
are ordered from left to right.

A \emph{bracket expression} of size $n$ is a sequence of $n$ opening brackets $\langle$ or closing
brackets $\rangle$.
It is a \emph{well-formed prefix} if, when read from left to right, the number of closing brackets
encountered never exceeds the number of opening brackets encountered.
A \emph{well-formed bracket expression} is a well-formed prefix with the same number of opening and
closing brackets.

It is a well-known result [TODO FIND A CITATION] that the number of well-formed bracket expressions
of size $2k$ is the Catalan number $C_k$. In fact, the recurrence can readily be seen from
a grammatical definition of well-formed bracket expressions,\footnote{We have not enclosed nonterminals
in angle brackets here, since that would result in hopeless confusion in a grammar describing nothing but
sequences of angle brackets.}\begin{align*}
\text{wfbe} \Coloneq
& \langle \text{wfbe} \rangle \text{wfbe} \\
\mid\: & \text{empty}\text,
\end{align*}
where the sum ranges over the sizes of the component bracket expressions.

Note that inserting an opening bracket, followed a closing bracket anywhere after the
inserted opening bracket, into a well-formed bracket expression, results in a well-formed bracket
expression.

Given a point set $P$  of size $n$ in general position with respect to the horizontal and a perfect
matching $\gm$ on $P$, any point in $P$ is either a left or right endpoint of an edge, since it
is incident to exactly one edge, and that edge is not vertical.

\marginfig[A well-formed bracket expression constructed from a perfect matching.]{
\begin{tikzpicture}[scale=0.5]
\tikzstyle{dot}=[draw,shape=circle,fill=black,scale=0.5]
\node[dot] at (0,2) (p1) {};
\node[dot] at (1,3) (p2) {};
\node[dot] at (2,4) (p3) {};
\node[dot] at (3,0) (p4) {};
\node[dot] at (5,1) (p5) {};
\node[dot] at (6,3) (p6) {};

\node at (0,-1) (b1) {$\langle$};
\node at (1,-1) (b2) {$\langle$};
\node at (2,-1) (b3) {$\rangle$};
\node at (3,-1) (b4) {$\langle$};
\node at (5,-1) (b5) {$\rangle$};
\node at (6,-1) (b6) {$\rangle$};

\draw (p1) -- (p5);
\draw (p2) -- (p3);
\draw (p4) -- (p6);

\draw[dashed] (p1) -- (b1);
\draw[dashed] (p2) -- (b2);
\draw[dashed] (p3) -- (b3);
\draw[dashed] (p4) -- (b4);
\draw[dashed] (p5) -- (b5);
\draw[dashed] (p6) -- (b6);
\end{tikzpicture}}
Construct a bracket expression $\gb\of\gm$ of size $n$ as follows: order $P$ from left to right;
the $i$th bracket is opening if the $i$th point of $P$ is a left endpoint of $\gm$, and it is
closing otherwise.
This bracket expression is well-formed, since it can be constructed by starting from the
(well-formed) empty bracket expression, by successively inserting both brackets corresponding to
each edge, where the closing bracket will be inserted to the right of the opening bracket.

Given a well-formed bracket expression $B$ and a point set $P$, we will say that a perfect matching
$\gm$ is \emph{consistent with} $B$ if $\gb\of\gm = B$; moreover, we will refer to the points of
$P$ corresponding to opening brackets of $B$ as \emph{left-points} (since they will be left endpoints
of any perfect matching consistent with $B$), and similarly we will refer to the points of $P$
corresponding to closing brackets of $B$ as \emph{right-points}.

One approach to bounding the size of $\CFPM_P$ is to bound the size of
$\gb^{-1}\of{B}\Intersection \CFPM_P$ for bracket expressions $B$ of size $n$\idest the
number of crossing-free perfect matchings on $P$ consistent with $B$. We thus define
\[\gn_P\of B \DefineAs \gb^{-1}\of{B}\Intersection \CFPM_P\text.\]
In order to concisely refer to bracket expressions, we will use the notations
$\langle^k$ for $k$ successive opening brackets, and $\rangle^k$ for $k$ successive
closing brackets, for instance,\[
\langle^2\rangle\langle^2\rangle^3 = \langle\langle\rangle\langle\langle\rangle\rangle\rangle
\text.\]

[TODO a section or subsection or something here, talking about the trivial upper bounds on $\langle\rangle$,
and about the bound on a product of bracket expressions. Maybe about the boring stuff like
$\langle\langle\rangle E \rangle$, but frankly that's not very interesting]

[TODO cite the existing upper bound from Sharir---Welzl 2006 (the proof is unrelated to anything here
though)]
\section{An optimal lower bound for the number of crossing-free perfect matchings}
This argument is due to E.~Welzl [TODO cite paper to appear, is there a preprint?].

\begin{theorem}[Welzl---maybe other people?]
Let $P$ be a point set of size $n = 2k$ in general position with respect to the horizontal,
and let $B$ be a well-formed bracket expression of size $n$.
Then there exists a crossing-free perfect matching consistent with $B$; in other words,
$\gn_P\of B \geq 1$.
\begin{proof}
Let $m_0$ be a perfect matching on $P$ consistent with $B$. This is always possible, for instance,
parsing the bracket expression, match the point corresponding to an opening parenthesis
and the point corresponding to the matching closing parenthesis.

Define $l(m)$ for a perfect matching $m$ on $P$ to be the sum of the lengths of the edges of $m$.

\marginfig[\label{figUncrossing}Uncrossing in a left-right perfect matching. Replacing the thick
edges by the thin ones reduces the total edge length, while preserving left and right endpoints.]{
\begin{tikzpicture}[scale=1]
\tikzstyle{dot}=[draw,shape=circle,fill=black,scale=0.5]
\node[dot,label=below:$a$] at (0,0) (a) {};
\node[dot,label=above:$c$] at (-0.5,1) (c) {};
\node[dot,label=below:$d$] at (2,0.5) (d) {};
\node[dot,label=above:$b$] at (1.5,1.5) (b) {};

\draw (a) -- (d);
\draw (c) -- (b);
\draw[very thick] (a) -- (b);
\draw[very thick] (c) -- (d);
\end{tikzpicture}}
Then, repeat the following procedure, starting at $i=0$.
If there is no crossing in $m_i$, we have found a perfect matching with the desired properties.
If there is a crossing, let $a$, $b$, $c$, and $d$ be the points involved, so that the edge
$ab$ crosses the edge $cd$. Remove these edges, and replace them by $ad$ and $cb$
(thus ``uncrossing'' them). This yields another perfect matching $m_{i+1}$. By the triangle
inequality (see figure \ref{figUncrossing}), $l(m_{i+1})<l(m_i)$.

If this did not terminate, it would yield a sequence $m$ of crossing perfect matchings on $P$ on
which $l$ is strictly decreasing, thus an infinite sequence of distinct graphs on $P$.
Since there are only finitely many graphs on $P$, this is a contradiction, so we eventually find a
crossing-free perfect matching.
\end{proof}
\end{theorem}
This immediately yields a lower bound for the number of crossing-free perfect matchings, since
there are $C_{\frac n 2}$ well-formed bracket expressions of size $n$.
\begin{corollary}
Let $P$ be a point set of size $n$ in general position. There are at least $C_{\frac n 2}$ distinct
crossing-free perfect matchings on $P$\idest $\Cardinality{\CFPM_P} \geq C_{\frac n 2}$.
\end{corollary}
Moreover, this lower bound is optimal, since it is attained if $P$ is in convex position.
\section{Matchings across a line}
Again we consider $n=2k$ points in general position with respect to the horizontal.
The left-right matchings corresponding to brackets expressions with $k$ opening brackets followed
by $k$ closing brackets, $\langle^k\rangle^k$, are called \emph{matchings
across a line}. Indeed, any edge in such a matching will cross any vertical line that separates
the left-points from the right-points.

The following result was shown by Micha Sharir and Emo Welzl in 2006 [TODO cite].
\begin{theorem}[Sharir--Welzl]
There are at most $C_{\frac n 2}^2$ crossing-free perfect matchings across a line on a point set $P$ of size
$n=2k$ points in general position with respect to the horizontal\idest
$\gn_P\of {\langle^k\rangle^k} < C_k^2$.
\begin{proof}
Pick a vertical line that separates the left-points from the right-points; we will call it \emph{the vertical line}.
Further, let us call set of left-points $L$ and the set of right-points $R$.

A perfect matching across a line is uniquely defined by a bijection $\FunctionSpec \gm L R$ from
the left-points to the right-points.
Now, number the intersections between the edges of the perfect matching and the vertical
line from top to bottom. This yield a numbering $\FunctionSpec \gi E {[k]}$ of the edges.

Define $\gl\of l \DefineAs \gi\of{e_l}$ mapping a left-point to the intersection number of its
edge, and similarly $\gr\of r \DefineAs \gi\of{e_r}$ for the right-points. We have $\gm = \gl \gr^{-1}$.

The bijection $\gl$ (respectively $\gr$) determines the order in which the left points (respectively right points)
reach the vertical line.

The idea of the proof is as follows: if the matching is crossing-free, we will show that
$\gl$ and $\gr$ have to be in sets of size $C_k$, thus that $\gm = \gl \gr^{-1}$ can take at most
$C_k^2$ values\idest that there can be at most $C_k^2$ perfect matchings across a line.

Since we are going to reuse these concepts in subsequent proofs, we will formalize and name the properties of
$\gl$ and $\gr$ that we will consider. A bijection from a set of left-points $\gL$ to $[\Cardinality \gL]$
that can be constructed by numbering from top to bottom the intersections of non-crossing edges incident to $\gL$ with a
vertical line to the right of $\gL$ is called a \emph{crossing-free left-alignment of $\gL$}. Correspondingly, for
right-points, we define a \emph{crossing-free right-alignment}.
\marginfig[A crossing-free left-alignment of five points. Once the index of the leftmost point is chosen (thick edge),
the rest consists in two crossing-free left-alignments of $i$ and $k-i-1$
points each---here $k = 5$ and $i = 2$.]{
\begin{tikzpicture}[scale=0.75]
\tikzstyle{dot}=[draw,shape=circle,fill=black,scale=0.5]
\node[dot] at (-5,2) (p1) {};
\node[dot] at (-4,1) (p2) {};
\node[dot] at (-3,3) (p3) {};
\node[dot] at (-2,1.5) (p4) {};
\node[dot] at (-1,3.5) (p5) {};

\node[inner sep=0pt,label=right:$1$] at (0,4) (l1) {};
\node[inner sep=0pt,label=right:$2$] at (0,3.5) (l2) {};
\node[inner sep=0pt,label=right:$3$] at (0,2.5) (l3) {};
\node[inner sep=0pt,label=right:$4$] at (0,1.5) (l4) {};
\node[inner sep=0pt,label=right:$5$] at (0,1) (l5) {};

\draw[->,very thick] (p1) -- (l3);
\draw[->] (p3) -- (l1);
\draw[->] (p5) -- (l2);
\draw[->] (p4) -- (l4);
\draw[->] (p2) -- (l5);

\draw[very thick] (0,5) -- (0,0);
\end{tikzpicture}}

If we have a crossing-free perfect matching across a line, then $\gl$ as constructed above is a crossing-free
left-alignment of $L$, and $\gr$ is a crossing-free right-alignment of $R$.

\begin{lemma}
There are at most $C_k$ crossing-free left-alignments of $k$ points.
\begin{proof}
Let $l$ be the leftmost point of a set of $L$ of $k$ left-points, and let $\gl$ be a crossing-free
left-alignment of $L$.

$\gl(l)$\idest the index of the crossing of the vertical line the edge $e_l$ of the leftmost point,
is equal to one plus the number of points that are above $e_l$. Indeed, the edges of points above $e_l$ must
themselves reach the vertical line above $e_l$, otherwise they would cross $e_l$, and correspondingly for
points below $e_l$, so that there are as many edges reaching the vertical line below $e_l$ as there are points
below $e_l$.

Moreover, as the oriented angle between $e_l$ and the horizontal increases, points are only added to the
set of points below $e_l$, so that choosing the number of points below $e_l$ determines the sets of points
below and above $e_l$.

Further, since points above $e_l$ must reach the vertical line above $e_l$, and correspondingly for points below,
if the point $p$ is above $e_l$, then $\gl(p) < \gl(l)$, and if it is below, $\gl(p) > \gl(l)$. Thus
$\gl$ restricted do the points above $e_l$ is a crossing-free left-alignment of the $\gl(l) - 1$ points above
$e_l$, and $\gl-\gl(l)$ is a crossing-free left-alignment of the $k - \gl(l)$ points below $e_l$.

It follows that $\gl$ is determined by the choice of $i \DefineAs \gl(l) - 1$ and crossing-free left-alignments
of $i$ and $k-i-i$ points. Therefore, if $\varpi_a$ is a bound for the number of crossing-free
left-alignments of $a$ points when $a < k$,
we can give a bound on the number of crossing-free left-alignments of $k$ points,
\[\varpi_k\DefineAs\sum{i=0}[k]\varpi_{i}\varpi_{k-i-1}\text.\]
We can start the recurrence with $\varpi_0 = 1$; this is the recurrence for the Catalan numbers, thus
$\varpi_k=C_k$.
\end{proof}
\end{lemma}
The same result holds for right-alignments, completing the proof of the theorem.
\end{proof}
\end{theorem}
\section{Analysing the overcounting in the upper bound for matchings across a line}
\section{Highly convex matchings across a line}
\section{Three changes of bracket direction}
We now consider left-right matchings corresponding to bracket expressions which have three
changes of bracket directions\idest bracket expressions of the form
$\langle^k\rangle^l\langle^q\rangle^p$,
$k$ opening brackets, $l$ closing brackets, $q$ opening brackets, $p$ closing brackets, where
$k-l=p-q\geq 0$ and $k+l+p+q\DefinitionOf n$.

In term of points, this means that four sets can be separated by vertical lines, from left
to right, $k$ left-points forming the set $K$, $l$ right-points forming $L$, $q$ left-points
forming $Q$, and $p$ right-points forming $P$.
We pick a vertical line separating $K$ and $L$ and call it \emph{the left line}, and we pick
a vertical line separating $P$ and $Q$ and call it \emph{the right line}.

Given a crossing-free perfect matching on those points, numbering from top to bottom the
intersections between edges incident to the points of $K$ and and the left line,
we get a crossing-free left-alignment of $K$. $k-l$ of the $k$ edges on this vertical
line are incident to points in $P$; the other $l$ are incident to points in $L$.
Numbering those $l$ edges yields a crossing-free right-alignment $\gl$ of $L$.

Similarly on the right side, we get a crossing-free right-alignment of $\gp$ of $P$,
numbering the subset of edges incident to a point in $P$ and a point in $Q$, we get a
left-alignment $\gq$ of $q$.
\begin{figure}[!ht]
\centering
\begin{tikzpicture}[scale=0.5]
\tikzstyle{dot}=[draw,shape=circle,fill=black,scale=0.25]
\node[dot] at (-9,1.9) (k1) {};
\node[dot] at (-8,1) (k2) {};
\node[dot] at (-7,4) (k3) {};
\node[dot] at (-6,3) (k4) {};
\node[dot] at (-5,0) (k5) {};

\node[dot] at (-3.5,1.5) (l1) {};
\node[dot] at (-2.5,0.5) (l2) {};
\node[dot] at (-1.5,2.5) (l3) {};
\node[dot] at (-0.5,3) (l4) {};

\node[dot] at (0.5,3.7) (p1) {};
\node[dot] at (1.5,0) (p2) {};
\node[dot] at (2.5,1) (p3) {};
\node[dot] at (3.5,2.8) (p4) {};

\node[dot] at (5,2.5) (q1) {};
\node[dot] at (6,0.5) (q2) {};
\node[dot] at (7,3.5) (q3) {};
\node[dot] at (8,1.5) (q4) {};
\node[dot] at (9,2.1) (q5) {};

\draw[very thick, name path=left line] (-4.25,5) -- (-4.25,-1);
\draw[very thick, name path=right line] (4.25,5) -- (4.25,-1);

\draw[gray, name path=edge] (k1) -- (q5);
\draw[->,thick,name intersections={of=edge and left line, by=align}] (k1) -- (align);
\draw[->,thick,name intersections={of=edge and right line, by=align}] (q5) -- (align);


\draw[gray, name path=edge] (k3) -- (l4);
\draw[->,thick,name intersections={of=edge and left line, by=align}] (k3) -- (align);
\draw[->,thick,name intersections={of=edge and left line, by=align}] (l4) -- (align);

\draw[gray, name path=edge] (k2) -- (l1);
\draw[->,thick,name intersections={of=edge and left line, by=align}] (k2) -- (align);
\draw[->,thick,name intersections={of=edge and left line, by=align}] (l1) -- (align);

\draw[gray, name path=edge] (k4) -- (l3);
\draw[->,thick,name intersections={of=edge and left line, by=align}] (k4) -- (align);
\draw[->,thick,name intersections={of=edge and left line, by=align}] (l3) -- (align);

\draw[gray, name path=edge] (k5) -- (l2);
\draw[->,thick,name intersections={of=edge and left line, by=align}] (k5) -- (align);
\draw[->,thick,name intersections={of=edge and left line, by=align}] (l2) -- (align);


\draw[gray, name path=edge] (q1) -- (p4);
\draw[->,thick,name intersections={of=edge and right line, by=align}] (q1) -- (align);
\draw[->,thick,name intersections={of=edge and right line, by=align}] (p4) -- (align);

\draw[gray, name path=edge] (q2) -- (p2);
\draw[->,thick,name intersections={of=edge and right line, by=align}] (q2) -- (align);
\draw[->,thick,name intersections={of=edge and right line, by=align}] (p2) -- (align);

\draw[gray, name path=edge] (q3) -- (p1);
\draw[->,thick,name intersections={of=edge and right line, by=align}] (q3) -- (align);
\draw[->,thick,name intersections={of=edge and right line, by=align}] (p1) -- (align);

\draw[gray, name path=edge] (q4) -- (p3);
\draw[->,thick,name intersections={of=edge and right line, by=align}] (q4) -- (align);
\draw[->,thick,name intersections={of=edge and right line, by=align}] (p3) -- (align);

\node at (-9,-1) {$\langle$};
\node at (-8,-1) {$\langle$};
\node at (-7,-1) {$\langle$};
\node at (-6,-1) {$\langle$};
\node at (-5,-1) {$\langle$};

\node at (-7,5) {$K$};

\node at (-3.5,-1) {$\rangle$};
\node at (-2.5,-1) {$\rangle$};
\node at (-1.5,-1) {$\rangle$};
\node at (-0.5,-1) {$\rangle$};

\node at (-2,5) {$L$};
\node at (2,5) {$Q$};

\node at (0.5,-1) {$\langle$};
\node at (1.5,-1) {$\langle$};
\node at (2.5,-1) {$\langle$};
\node at (3.5,-1) {$\langle$};

\node at (7,5) {$P$};

\node at (5,-1) {$\rangle$};
\node at (6,-1) {$\rangle$};
\node at (7,-1) {$\rangle$};
\node at (8,-1) {$\rangle$};
\node at (9,-1) {$\rangle$};
\end{tikzpicture}
\caption{Four crossing-free alignments.\label{figCFPMklqp}}
\end{figure}

The matching is uniquely determined by $\gk$, $\gp$, the choice of
the $k-l$ among $k$ points and $k-l=p-q$ among $p$ points that get matched to each other,
and by $\gl$ and $\gq$, which gives the following bound for the number of these matchings:\[
\gn_P\of{\langle^k\rangle^l\langle^q\rangle^p}
\leq C_k \binom k l C_l C_q \binom p q C_p\text.\]
Asymptotically, the factor involving $k$ and $l$ is\[
C_k \binom k l C_l
\preccurlyeq
4^{k+l} \binom k l
\preccurlyeq
4^{k+l}
\pa{
\pa{\frac
  {\pa{1-\ga}^{1-\frac 1 \ga}}
  {\ga}}^{\pa{\frac{\ga}{\ga+1}}}}^{k+l}\text,
\]
where $l=\ga k$. The base of the exponential bound for the binomial coefficient is maximal
when $\ga = \frac{3-\sqrt{5}}{2}$, where it is $\GoldenRatio = \frac{1+\sqrt{5}}{2}$.
This gives the overall bound\[
\gn_P\of{\langle^k\rangle^l\langle^q\rangle^p}
\preccurlyeq
4^{k+l}
\GoldenRatio^{k+l}
4^{p+q}
\GoldenRatio^{p+q}
= \pa{4\GoldenRatio}^n
\approx 6.472^n\text.
\]
\subsection{Improving the binomial bound}
We can however improve upon that bound: indeed, once $\gk$, $\gp$, and the set of points
of $K$ matched to $P$ are fixed, the edges of the matching that cross both the left line
and the right line---let us call these \emph{long edges}---are determined. As a result,
the region between the left line and the
right line is partitioned in trapezoidal cells, and the portion of any edge from $K$ to $L$ and
from $P$ to $Q$ that lies between the left line and the right line is confined to a
single of those cells. It follows that $\gl$ is composed of crossing-free
right-alignments the of subsets of $L$ separated by the long edges, and similarly for $\gq$
with subsets of $Q$ (in figure~\ref{figCFPMklqp}, there is one long edge, and thus two cells).

Let us look at the edges crossing the left line (the same argument applies to the right line),
numbered from top to bottom: $k-l$ of those are
long edges, let $S \Subset [k]$ be their numbers; in between two long edges, above the first
long edge, and below the last one, we have the edges that define the crossing-free right-alignments
that make up $\gl$. It follows that a crossing-free right-alignment of $m$ points that makes up
$\gl$ corresponds to a maximal sequence of $m$ consecutive elements of $[k] \setminus S$.

We will call the set of maximal sequences of consecutive elements of $S'$ the \emph{cells} of
$S'$, written $\cells\of{S'}$.

Then, we can improve the $\binom k l C_l$ factor in the bound (in which the binomial comes from
the choice of the long edges amongst the $k$ edges on the left line, and the Catalan number comes
from the choice of $\gl$), summing over the choices of the long edges (and thus of $S$ above).
The improved factor becomes\begin{equation}
\spc\of{k, l}\DefineAs
\sum{S\in\binom{[k]}{k-l}} \quad \prod{c\in\cells\of{[k] \setminus S}}C_{\Cardinality c} =
\sum{S'\in\binom{[k]}{l}} \quad \prod{c\in\cells\of{S'}}C_{\Cardinality c}\text,
\label{spcDefinition}
\end{equation}
and the overall bound becomes\begin{equation}
\gn_P\of{\langle^k\rangle^l\langle^q\rangle^p}\leq
C_k\spc\of{k, l}\spc\of{p, q}C_p\text.\label{bound-ck-spckl-spcpq-cp}
\end{equation}

\subsection{A recurrence}
In order to compute $\spc$ efficiently, and eventually, get its asymptotics, it is useful to get rid
of the $\cells$ function. We can express $\spc$ as a recurrence instead.
First, we note that $\spc\of{k,k} = C_k$: there is only one summand, $S'$ is the whole set, so
it has only one cell, namely itself.
Otherwise, $k-l\geq 1$; in the sum over the $S$, consider the greatest element $j$ of $S$, which is at least
$k-l$, and split the sum over that,
\[\spc\of{k, l} = \sum{j=k-l}[k]\quad
\sum{\substack{S\in\binom{[k]}{k-l} \\ j=\max S}} \quad
\prod{c\in\cells\of{[k] \setminus S}}C_{\Cardinality c}\text.
\]
For fixed $j$, all summands (of the sum over $S$) will have a factor
with $c=\set{j+1,\dotsc,k}$, and thus a factor of
$C_{k-j}$. Factoring out this $C_{k-j}$, we get
\[\spc\of{k, l} = \sum{j=k-l}[k]
C_{k-j} \quad
\sum{\substack{S\in\binom{[k]}{k-l} \\ j=\max S}} \quad
\prod{\substack{c\in\cells\of{[k] \setminus S} \\ c\neq\set{j+1,\dotsc,k}}}
    C_{\Cardinality c}\text.
\]
Now, note that choosing a subset $S$ of $[k]$ of size $k-l$ whose maximum is $j$ is equivalent to choosing
a subset $s$ of $[j-1]$ of size $k-l - 1$, where $S=s\Union\set{j}$. Moreover, the cells of $[k]\setminus S$ other
than $\set{j+1,\dotsc,k}$ are exactly the cells of $[j-1]\setminus s$, thus
\[\spc\of{k, l} = \sum{j=k-l}[k]
C_{k-j} \quad
\sum{s\in\binom{[j-1]}{k-l - 1}} \quad
\prod{c\in\cells\of{[j-1]\setminus s}}
    C_{\Cardinality c}\text.
\]
By definition of $\spc$, this means
\[\spc\of{k, l} = \sum{j=k-l}[k]
C_{k-j}
\spc\of{j-1,l+j-k}\text.
\]
Rewriting this as a sum over $i\DefineAs k-j$, this gives us the following recurrence for $\spc$:
\begin{align}\spc\of{k, l} &= \sum{i=0}[l]
C_{i}
\spc\of{k-i-1,l-i} &\text{for $l<k$,} \label{spcRecurrence1}\\
\spc\of{k, k} &= C_k \label{spckk}\text.
\end{align}\marginfig[The first few values of $\spc$; $k$ vertically from $0$ to $5$,
$l$ horizontally from $0$ to $k$. The $l+1$ summands in the recurrence (\ref{spcRecurrence1})
with $k = 6$ and $l = 3$ are highlighted.]{
$\begin{array}{r r r r r r}
1 \\
{\color{red}1} & 1 \\
1 & {\color{red}2} &  2 \\
1 & 3 &  {\color{red}5} & 5 \\
1 & 4 &  9 & \tn{lastSummand}{\color{red}$14$} & 14 \\
1 & 5 & 14 & \tn{result}{$28$} & 42 & 42
\end{array}$
\tikz[remember picture,overlay]
    \path[->] (lastSummand.south) edge (result.north -| lastSummand.south);
}
\subsection{A better recurrence}
We can now turn this recurrence into a simpler recurrence, which we will prove by recurrence.

Since $\spc$ has not been formally defined for negative arguments, we extend the definition
with $\spc\of{k,l}=0$ for $l<0$; this is consistent with the definition, since it yields a sum over
subsets $S\Subset [k]$ bigger than $k$, and it yields an empty sum in the recurrence we just derived.
In addition to that, we also let $\spc\of{k,l} = 0$ when $l > k$.

Note that for $k - 1 = l\geq 0$, we have\begin{align}
\spc\of{k, l}
= \sum{i=0}[l]C_{i}\spc\of{k-i-1,l-i} = \sum{i=0}[l]C_{i}\spc\of{l-i,l-i} 
&= \sum{i=0}[l]C_{i}C_{l-i} \nonumber \\
&= C_{l+1}\text,\label{spckkminusone}
\end{align}
and thus 
\begin{equation}
\spc\of{k, l} = \spc\of{k-1, l} + \spc\of{k, l-1} = C_k \text{ for $k = l > 0$.}
\end{equation}
Further, for $k - 1 = l$, we get
\begin{align}
\spc\of{k, l}
&= \sum{i=0}[l]C_{i}C_{l-i} = C_l + \sum{i=0}[l-1]C_{i}C_{((l-1)-i)+1} \nonumber\\
\intertext{applying (\ref{spckk}) on the left and (\ref{spckkminusone}) on the right,}
&= \spc\of{k-1, l} + \sum{i=0}[l-1]C_{i}\spc\of{k-i-1,(l-1)-i} \nonumber\\
&= \spc\of{k-1, l} + \spc\of{k,l-1}\text.
\end{align}

Now let $k - 1 > l \geq 0$.
Assume $\spc\of{k',l'}=\spc\of{k'-1,l'}+\spc\of{k',l'-1}$ for $0\leq l'<k'<k$.
Then we can apply this assumption to the summands of $\spc\of{k, l}$:
\begin{align}
\spc\of{k, l} &= \sum{i=0}[l]
C_{i}
\spc\of{k-i-1,l-i} \nonumber \\
&= \sum{i=0}[l]
C_{i}
\pa{\spc\of{k-i-2,l-i}+\spc\of{k-i-1,l-i-1}} \nonumber \\
&= \sum{i=0}[l]
C_{i}\spc\of{k-i-2,l-i}
+\sum{i=0}[l]C_{i}\spc\of{k-i-1,l-i-1} \nonumber \\
&= \sum{i=0}[l]
C_{i}\spc\of{(k-1)-i-1,l-i}
+\sum{i=0}[l-1]C_{i}\spc\of{k-i-1,(l-1)-i} \\&\phantom{=}\qquad + C_{l}\spc\of{k-l-1,-1}\text, \nonumber \\
\intertext{so, substituting the recurrence for $\spc$,}
&= \spc\of{k-1, l} + \spc\of{k,l-1}\text.
\end{align}
We thus have the following recurrence for $\spc$:
\begin{align}
\spc\of{k,l} &= \spc\of{k-1, l} + \spc\of{k,l-1} & \text{for $k\geq 1$, $0\leq l \leq k$,}\label{spc-catalan-triangle}\\
\spc\of{k,l} &= 0 & \text{for $l < 0$ or $l > k$,}\\
\spc\of{0,0} &= 1\text.
\end{align}\marginfig[The recurrence (\ref{spc-catalan-triangle}).]{
$\begin{array}{r r r r r r}
1 \\
1 & 1 \\
1 & 2 &  2 \\
1 & 3 &  5 & 5 \\
1 & 4 &  9 & \tn{summandAbove}{$14$} & 14 \\
1 & 5 & \tn{summandLeft}{$14$} & \tn{result}{$28$} & 42 & 42
\end{array}$
\tikz[remember picture,overlay]
    \path[->] (summandAbove.south) edge (result.north -| summandAbove.south)
              (summandLeft) edge (result);
}
This is the recurrence defining the Catalan triangle [CITATIONS], and its solution is known, namely [CITATION]
\begin{equation}
\spc\of{k,l} = \frac
{\Factorial{\pa{k+l}} \pa{k-l+1}}
{\Factorial l \Factorial{\pa{k+1}}}
= \frac{k-l+1}{k+1} \binom{k+l}{k}\label{spc-solution}
\text.
\end{equation}
\subsubsection{A combinatorial interpretation of $\spc$}
The Catalan triangle counts well-formed prefix bracket expressions [CITATION]. This can in fact be seen from
all the definitions of $\spc$ above, thus providing a more combinatorial solution.

In the original definition (\ref{spcDefinition}) of $\spc$, the sum is over the choices for the positions of the
unmatched $k-l$ opening brackets; between those, well-formed bracket expressions (counted by Catalan numbers) are
inserted.

The first recurrence (\ref{spcRecurrence1}) corresponds to the following grammatical
definition\footnote{This grammar requires infinite look-ahead to parse, but doing a first pass to mark the
unmatched opening brackets resolves that.} of well-formed
prefixes, where again wfbe denotes a well-formed bracket expression:
\begin{align*}
\text{wfprefix} \Coloneq
& \text{wfbe} \langle \text{wfprefix} \\
\mid\: & \text{wfbe},
\end{align*}
where the sum ranges over the size of the well-formed bracket expression preceding the first unmatched bracket.

Finally, (\ref{spc-catalan-triangle}) reflects the fact that a well-formed prefix with $k$ opening brackets
and $l$ closing brackets either ends with an opening bracket (preceded by a well-formed prefix with $k-1$ opening
and $l$ closing brackets), or with a closing bracket (preceded by a well-formed prefix with $k$ opening
and $l-1$ closing brackets).
\subsection{Asymptotics}
We can now use the expression (\ref{spc-solution}) for $\spc$ to study the asymptotics of the bound
(\ref{bound-ck-spckl-spcpq-cp}).
We are interested in the asymptotics as a function of the size of of the bracket expression $n = k+l+p+q$.
Since the bound is a product of two identical two-parameter factors, we study one of them,\[
C_k\spc\of{k,l}\text,
\]
as a function of $n_1\DefineAs k+l$. Let $l = \ga k$, thus $0\leq \ga\leq 1$,
the above expression becomes \[
C_{\frac{n_1}{\ga + 1}} \binom{n_1}{\frac{n_1}{\ga + 1}} \frac{\pa{1-\ga}n+\ga+1}{n+\ga+1}\text.
\]
Asymptotically as $n_1\longrightarrow \infty$, this yields\[
\frac
  {1}
  {n_1\sqrt{\Pi n_1}}
4^{\frac{n_1}{\ga + 1}}
\frac
  {\sqrt{1+\ga}}
  {\sqrt{2\Pi n_1 \ga}}
\pa{
  \frac
    {1+\ga}
    {\ga^{\frac{\ga}{\ga + 1}}}}^{n_1}
\pa{1-\ga}\text,
\]
or, up to a polynomially-bounded factor,\[
4^{\frac{n_1}{\ga + 1}}
\pa{
  \frac
    {1+\ga}
    {\ga^{\frac{\ga}{\ga + 1}}}}^{n_1}
=
\pa{
4^{\ga + 1}
\frac
  {1+\ga}
  {\ga^{\frac{\ga}{\ga + 1}}}}^{n_1}\text.
\]
As $\ga$ ranges from $0$ to $1$, the base of that exponential reaches a maximum of $5$ at
$\ga=\frac 1 4$. We thus have the following asymptotic bound.\begin{equation}
\gn_P\of{\langle^k\rangle^l\langle^q\rangle^p}\leq
C_k\spc\of{k, l}\spc\of{p, q}C_p
\preccurlyeq
5^{k+l}5^{p+q} = 5^n
\text.
\end{equation}
[TODO: would it be interesting or feasible to average over all $\langle^k\rangle^l\langle^q\rangle^p$,
for fixed $n$?]
\clearpage
\nocite{*}
\bibliography{crossing-free-perfect-matchings}
\bibliographystyle{plain}

\end{document}
